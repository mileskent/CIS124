\documentclass[10pt, a4paper, twocolumn]{article}

% \usepackage[english]{babel} % English language hyphenation

% \usepackage{microtype} % Better typography

\usepackage{ocgx2} %implements PDF Layers

\usepackage{hyperref}

\usepackage{float}

\usepackage{amsmath,amsfonts,amsthm} % Math packages for equations

\usepackage[svgnames]{xcolor} % Enabling colors by their 'svgnames'
\hypersetup{
    colorlinks,
    linkcolor={red!50!black},
    citecolor={blue!50!black},
    urlcolor={blue!80!black}
}

\usepackage[hang, small, labelfont=bf, up, textfont=it]{caption} % Custom captions under/above tables and figures

\usepackage{booktabs} % Horizontal rules in tables

\usepackage{lastpage} % Used to determine the number of pages in the document (for "Page X of Total")

\usepackage{graphicx} % Required for adding images

\usepackage{enumitem} % Required for customising lists
\setlist{noitemsep} % Remove spacing between bullet/numbered list elements

% \usepackage{sectsty} % Enables custom section titles
%\allsectionsfont{\usefont{OT1}{phv}{b}{n}} % Change the font of all section commands (Helvetica)

%----------------------------------------------------------------------------------------
%	MARGINS AND SPACING
%----------------------------------------------------------------------------------------

\usepackage{geometry} % Required for adjusting page dimensions

\geometry{
	top=1cm, % Top margin
	bottom=1.5cm, % Bottom margin
	left=2cm, % Left margin
	right=2cm, % Right margin
	includehead, % Include space for a header
	includefoot, % Include space for a footer
	%showframe, % Uncomment to show how the type block is set on the page
}

\setlength{\columnsep}{7mm} % Column separation width

%----------------------------------------------------------------------------------------
%	FONTS
%----------------------------------------------------------------------------------------

% \usepackage[T1]{fontenc} % Output font encoding for international characters
% \usepackage[utf8]{inputenc} % Required for inputting international characters

%----------------------------------------------------------------------------------------
%	HEADERS AND FOOTERS
%----------------------------------------------------------------------------------------

\usepackage{fancyhdr} % Needed to define custom headers/footers
\pagestyle{fancy} % Enables the custom headers/footers

\renewcommand{\headrulewidth}{0.0pt} % No header rule
\renewcommand{\footrulewidth}{0.4pt} % Thin footer rule

\renewcommand{\sectionmark}[1]{\markboth{#1}{}} % Removes the section number from the header when \leftmark is used

%\nouppercase\leftmark % Add this to one of the lines below if you want a section title in the header/footer

% Headers
\lhead{} % Left header
\chead{\textit{\thetitle}} % Center header - currently printing the article title
\rhead{} % Right header

% Footers
\lfoot{} % Left footer
\cfoot{} % Center footer
\rfoot{\footnotesize Page \thepage\ of \pageref{LastPage}} % Right footer, "Page 1 of 2"

\fancypagestyle{firstpage}{ % Page style for the first page with the title
	\fancyhf{}
	\renewcommand{\footrulewidth}{0pt} % Suppress footer rule
}

%----------------------------------------------------------------------------------------
%	TITLE SECTION
%----------------------------------------------------------------------------------------

\newcommand{\authorstyle}[1]{{\large\usefont{OT1}{phv}{b}{n}\color{DarkRed}#1}} % Authors style (Helvetica)

\newcommand{\institution}[1]{{\footnotesize\usefont{OT1}{phv}{m}{sl}\color{Black}#1}} % Institutions style (Helvetica)

\usepackage{titling} % Allows custom title configuration

\newcommand{\HorRule}{\color{DarkRed}\rule{\linewidth}{1pt}} % Defines the gold horizontal rule around the title

\pretitle{
	\vspace{-30pt} % Move the entire title section up
	\HorRule\vspace{10pt} % Horizontal rule before the title
	\fontsize{32}{36}\usefont{OT1}{phv}{b}{n}\selectfont % Helvetica
	\color{DarkRed} % Text colour for the title and author(s)
}

\posttitle{\par\vskip 15pt} % Whitespace under the title

\preauthor{} % Anything that will appear before \author is printed

\postauthor{ % Anything that will appear after \author is printed
	\vspace{10pt} % Space before the rule
	\par\HorRule % Horizontal rule after the title
	\vspace{20pt} % Space after the title section
}

%----------------------------------------------------------------------------------------
%	ABSTRACT
%----------------------------------------------------------------------------------------

\usepackage{lettrine} % Package to accentuate the first letter of the text (lettrine)
\usepackage{fix-cm}	% Fixes the height of the lettrine

\newcommand{\initial}[1]{ % Defines the command and style for the lettrine
	\lettrine[lines=3,findent=4pt,nindent=0pt]{% Lettrine takes up 3 lines, the text to the right of it is indented 4pt and further indenting of lines 2+ is stopped
		\color{DarkRed}% Lettrine colour
		{#1}% The letter
	}{}%
}

\usepackage{xstring} % Required for string manipulation

\newcommand{\lettrineabstract}[1]{
	\StrLeft{#1}{1}[\firstletter] % Capture the first letter of the abstract for the lettrine
	\initial{\firstletter}\textbf{\StrGobbleLeft{#1}{1}} % Print the abstract with the first letter as a lettrine and the rest in bold
}



\hbadness=10000
% TITLE PAGE
\title{CIS 124: Intro to Data Science}
\author{\authorstyle{Miles Kent}}
\date{}
\begin{document}
\graphicspath{ {./images/} }

\maketitle


    Data Visualization, Causality and Experiments\\
    Python Expression, Strings and Table methods\\
    Charts and Histograms\\
    Probability and Iteration, Sampling, Simulation\\
    Hypothesis Testing and Models\\
    Data interpretation related to the above\\

    \section{Section 2}
    Observation vs Experiment\\
    What constitutes causation?\\

    \section{Section 3}
    Basic Python \& its conventions\\
    \subsection{Table().select()}
    tbl.select("Col1", "Col2", ...)
    \subsection{Table().sort()}
    tbl.sort("Col", descending=$\textbf{BOOL}$)\\
    descending = True $\rightarrow$ big to small\\

    \section{Section 4}
    \subsection{make\_array(...)}
    Not really a vector. Element wise operations, i.e. when you multiply two 1D arrays it doesn't dot them, it multiplies each corresponding element and gives the resultant array. It's more of a table column except it's not in a Table() yet.\\
    make\_array(1, 2, 3, 4)\\
    \subsection{Table().column(...)}
    Table().column(nameorindex)\\
    Outputs array\\


    \section{Section 5}
    Table().num\_rows\\
    Table().num\_columns\\
    Table().labels\\
    \subsection{Table().relabled(...)}
    Table().relabeled("col1name", "NEWNAME")
    \subsection{Table.read\_table(...)}
    Table.read\_table("filename.csv")\\
    \subsection{Table().with\_columns(...)}
    Table().with\_columns("col1name", col1arr, ...)\\
    \subsection{Table().drop(...)}
    Table().drop("col1name", ...)\\
    \subsection{Table().take(...)}
    Table().take(rowindicestouse)\\
    \subsection{Table().where(...)}
    Table().take("colname", condition)\\
    condition uses "are" syntax.\\
    are.equal\_to()\\
    are.above()\\
    are.above\_or\_equal\_to()\\
    are.below()\\
    are.below\_or\_equal\_to()\\
    are.between()\quad [A, B)\\
    are.between\_or\_equal\_to()\quad [A, B)\\
    are.contained\_in()\\
    are.containing()\\
    are.strictly\_between()\quad (A, B)\\

    \section{Section 6}
    \subsection{Table().barh(...)}
    Table().barh(yaxiscategoriescolumn)
    Table().barh(yaxiscategoriescolumn, values)
    \subsection{Table().show(...)}
    Table().show(num)

    \section{Section 7}
    \subsection{Table().scatter(...)}
    Table().scatter(xcol, ycol, fit\_line=BOOL)\\
    \subsection{Table.plot(...)}
    Table().plot(xcol, ycol)\\
    Similar to scatter but it connects the dots\\

    \section{Section 8}
    \subsection{Table().hist(...)}
    Table().hist("colname", bins=make\_array(...))\\

    \section{Section 9}
    \subsection{Table().apply(...)}
    Table().apply(functionname, "columnname")\\

    \section{Section 10}
    \subsection{Table().group(...)}
    Table().group(columnname(s))\\
    Table().group(columnname(s), func)\\
    Rows are grouped by unique values. Func would be something that takes an arrat and returns a single value.
    \subsection{Table.pivot(...)}
    Table.pivot("col1name", "col2name", ...)\\
    Table.pivot("col1name", "col2name", ..., values="colXname", collect=func)\\
    Col1 is x categories, Col2 is y categories, values is the data points in each box, collect is the function that is applied to each categorical data point, e.g. collect=sum would give you the sum for each one.

    \section{Section 11}
    \subsection{Table().join(...)}
    tbl1.join("colthingfrom1", tbl2, "colsamethingfrom2")
    Joins two tables based on a like category/column.

    \section{Section 12}
    Yawn...\\

    \section{Section 13}
    Loops and also you can iterate over make\_array(...) btw.\\

    \section{Section 14}
    P(Both A and B) = P(A) * P(B) if independent\\
    If A can only happen in one of two possibilities, P(A) = P(A1) + P(A2)\\


    \section{Section 15}
    Deterministic vs Probability vs Uniform Random Sample\\
    $textbf{Deterministic:}$ you just choose some values for a sample. For example, I choose 3, 18, and 100 for my sample.\\
    $textbf{Probability:}$ a sample for which you can calculate the probability of any subset being the sample before the sample has been drawn.\\
    $textbf{Uniform Random:}$ randomly chosen.\\
    Also understand the law of large numbers.\\
    Whatever vairation distance is.
    \section{Section 16-18}
    Stats stuff.\\

    \section{Section 19}
    \subsection{Table().sample(...)}
    Table.sample(nrows, with\_replacement=BOOL)\\

\end{document}

