\documentclass[10pt, a4paper, twocolumn]{article}

\input{structure.tex}
\hbadness=10000
% TITLE PAGE
\title{CIS 124: Intro to Data Science}
\author{\authorstyle{Miles Kent}}
\date{}
\begin{document}
\graphicspath{ {./images/} }

\maketitle


    Data Visualization, Causality and Experiments\\
    Python Expression, Strings and Table methods\\
    Charts and Histograms\\
    Probability and Iteration, Sampling, Simulation\\
    Hypothesis Testing and Models\\
    Data interpretation related to the above\\

    \section{Section 2}
    Observation vs Experiment\\
    What constitutes causation?\\

    \section{Section 3}
    Basic Python \& its conventions\\
    \subsection{Table().select()}
    tbl.select("Col1", "Col2", ...)
    \subsection{Table().sort()}
    tbl.sort("Col", descending=$\textbf{BOOL}$)\\
    descending = True $\rightarrow$ big to small\\

    \section{Section 4}
    \subsection{make\_array(...)}
    Not really a vector. Element wise operations, i.e. when you multiply two 1D arrays it doesn't dot them, it multiplies each corresponding element and gives the resultant array. It's more of a table column except it's not in a Table() yet.\\
    make\_array(1, 2, 3, 4)\\
    \subsection{Table().column(...)}
    Table().column(nameorindex)\\
    Outputs array\\


    \section{Section 5}
    Table().num\_rows\\
    Table().num\_columns\\
    Table().labels\\
    \subsection{Table().relabled(...)}
    Table().relabeled("col1name", "NEWNAME")
    \subsection{Table.read\_table(...)}
    Table.read\_table("filename.csv")\\
    \subsection{Table().with\_columns(...)}
    Table().with\_columns("col1name", col1arr, ...)\\
    \subsection{Table().drop(...)}
    Table().drop("col1name", ...)\\
    \subsection{Table().take(...)}
    Table().take(rowindicestouse)\\
    \subsection{Table().where(...)}
    Table().take("colname", condition)\\
    condition uses "are" syntax.\\
    are.equal\_to()\\
    are.above()\\
    are.above\_or\_equal\_to()\\
    are.below()\\
    are.below\_or\_equal\_to()\\
    are.between()\quad [A, B)\\
    are.between\_or\_equal\_to()\quad [A, B)\\
    are.contained\_in()\\
    are.containing()\\
    are.strictly\_between()\quad (A, B)\\

    \section{Section 6}
    \subsection{Table().barh(...)}
    Table().barh(yaxiscategoriescolumn)
    Table().barh(yaxiscategoriescolumn, values)
    \subsection{Table().show(...)}
    Table().show(num)

    \section{Section 7}
    \subsection{Table().scatter(...)}
    Table().scatter(xcol, ycol, fit\_line=BOOL)\\
    \subsection{Table.plot(...)}
    Table().plot(xcol, ycol)\\
    Similar to scatter but it connects the dots\\

    \section{Section 8}
    \subsection{Table().hist(...)}
    Table().hist("colname", bins=make\_array(...))\\

    \section{Section 9}
    \subsection{Table().apply(...)}
    Table().apply(functionname, "columnname")\\

    \section{Section 10}
    \subsection{Table().group(...)}
    Table().group(columnname(s))\\
    Table().group(columnname(s), func)\\
    Rows are grouped by unique values. Func would be something that takes an arrat and returns a single value.
    \subsection{Table.pivot(...)}
    Table.pivot("col1name", "col2name", ...)\\
    Table.pivot("col1name", "col2name", ..., values="colXname", collect=func)\\
    Col1 is x categories, Col2 is y categories, values is the data points in each box, collect is the function that is applied to each categorical data point, e.g. collect=sum would give you the sum for each one.

    \section{Section 11}
    \subsection{Table().join(...)}
    tbl1.join("colthingfrom1", tbl2, "colsamethingfrom2")
    Joins two tables based on a like category/column.

    \section{Section 12}
    Yawn...\\

    \section{Section 13}
    Loops and also you can iterate over make\_array(...) btw.\\

    \section{Section 14}
    P(Both A and B) = P(A) * P(B) if independent\\
    If A can only happen in one of two possibilities, P(A) = P(A1) + P(A2)\\


    \section{Section 15}
    Deterministic vs Probability vs Uniform Random Sample\\
    $textbf{Deterministic:}$ you just choose some values for a sample. For example, I choose 3, 18, and 100 for my sample.\\
    $textbf{Probability:}$ a sample for which you can calculate the probability of any subset being the sample before the sample has been drawn.\\
    $textbf{Uniform Random:}$ randomly chosen.\\
    Also understand the law of large numbers.\\
    Whatever vairation distance is.
    \section{Section 16-18}
    Stats stuff.\\

    \section{Section 19}
    \subsection{Table().sample(...)}
    Table.sample(nrows, with\_replacement=BOOL)\\

\end{document}

